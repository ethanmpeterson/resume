\documentclass{resume} % Use the custom resume.cls style

\usepackage[left=0.75in,top=0.6in,right=0.75in,bottom=0.6in]{geometry} % Document margins
\usepackage{hyperref}

\name{Ethan Peterson}
\address{647 269 7873 \\ ethan@petetech.net}
\address{
  \href{http://portfolio.petetech.net}{Projects} \\
  \href{https://github.com/ethanmpeterson}{GitHub} \\
  \href{https://www.linkedin.com/in/ethanmpeterson}{LinkedIn}
}
\address{32 Roxborough St. E M4W 1V6 Toronto, Ontario, Canada}

\begin{document}

%----------------------------------------------------------------------------------------
%	EDUCATION SECTION
%----------------------------------------------------------------------------------------

\begin{rSection}{Education}

\begin{rSubsection}{Queen's University}{September 2018 - May 2023}{Intended
    Designation: BASc Computer Engineering (Entering 4th Year)}{Kingston, Ontario}
\item Dean's Scholar, awarded to students with a GPA of 3.5 or higher (1st, 
  2nd \& 3rd year)
\item Principal's Entrance Scholarship (2018)
\item GPA: 4/4.3
\end{rSubsection}
\end{rSection}

%----------------------------------------------------------------------------------------
%	WORK EXPERIENCE SECTION
%----------------------------------------------------------------------------------------

\begin{rSection}{Experience}

\begin{rSubsection}{Kepler Communications}{May 2021 - Present}{Hardware
    Engineering Intern}{Toronto, Ontario}
    % \item Working with full time engineers in upcoming satellite designs.
    \item Designed hardware and software for a motor controller PCB.
    \item Owned the requirements capture, scoping, schematic, layout and release of a flight PCB design. This design includes several high speed Aurora lanes, 10G Ethernet and distributes over a kilowatt of power.
    \item Continuing to contribute PCB designs that support Kepler's network infrastructure and overarching goal of putting the internet in space.
\end{rSubsection}

\begin{rSubsection}{Kepler Communications}{May 2020 - August 2020}{Hardware
    Test Engineering Intern}{Toronto, Ontario}
\item Worked with full time engineers to bring the company's next generation of
  low earth orbit (LEO) satellites to fruition.
\item Designed a selection of custom PCBs to support the satellite manufacturing
  process. Designs include, but are not limited to, internet enabled JTAG
  debuggers for FPGAs, flight sensor calibration PCBs, TVAC capable wiring
  harnesses and power boards with slew rate control.
\item Debugged flight boards using an oscilloscope with voltage
  and current probes.
\item Wrote the Python and C code required to test each PCB design as well as
  debugging existing scripts.
\end{rSubsection}

%------------------------------------------------

\begin{rSubsection}{Wattpad Inc.}{May 2019 - August 2019}{Associate
    Engineer}{Toronto, Ontario}
\item Worked with the Velocity squad on a variety of internal tools for Wattpad engineering.
  Namely \href{https://github.com/Wattpad/ship-it}{Ship-it!}, an open source continuous deployment tool running atop Kubernetes and
  Helm.
\item Wrote the Ship-it! frontend dashboard using React and MaterialUI.
\item Built a synchronization package in GoLang to reconcile the state of the
  Kubernetes Cluster and the deployment specifications for each service in Git.
\end{rSubsection}

%------------------------------------------------

\begin{rSubsection}{Marrelli Support Services Inc.}{June 2018 - August
    2018}{Software Developer}{Toronto, Ontario}
\item Responsible for developing custom software for a firm specialized in providing corporate
  services to publicly-traded companies.
\item Developed a custom volatility calculator to be used by the firm to calculate the historical
  volatility of equities traded on the TSX and TSX Venture Exchange.
\item Liaised with members of the firm’s upper management to independently design and
  develop a custom payment tracking software to assist the firm in its transition to a paperless
  bookkeeping process. The software complies with the firm’s internal control procedures,
  allows for communication between departments and guides users through the process so as
  to minimize bookkeeping errors.
\end{rSubsection}
%------------------------------------------------
\begin{rSubsection}{Braincubator}{June 2017 - August 2017}{Backend
    Developer}{Toronto, Ontario}
\item Summer internship with a start-up company specialized in delivering stem-based
  workshops.
\item Programmed the backend of the company website using Python and developed a Microsoft
  Kinect demo using Java.
\end{rSubsection}
%------------------------------------------------

\end{rSection}

%----------------------------------------------------------------------------------------
%	TECHNICAL STRENGTHS SECTION
%----------------------------------------------------------------------------------------

\begin{rSection}{Technical Skills}

\begin{tabular}{ @{} >{\bfseries}l @{\hspace{6ex}} l }
Office Software & Word, Powerpoint, Excel, OneNote and \LaTeX\\ \\

Hardware & Oscilloscope Operation, SPI, I2C, USART, CAN, Altium Designer, EAGLE, KiCAD, \\ & Schematic Design, PCB Layout, and STM32 \\ \\

Programming & Java, JavaScript (React and JSX), Python, Swift, C, GoLang, AVR Assembly,\\ & Bash Script, Linux, SVN, Git, Kubernetes, and Amazon Web Services. \\


\end{tabular}

\end{rSection}


%----------------------------------------------------------------------------------------
%	EXAMPLE SECTION
%----------------------------------------------------------------------------------------

%	Volunteering Activities and Projects
%----------------------------------------------------------------------------------------
\begin{rSection}{Extracurricular Experience}


\begin{rSubsection}{Queen's Formula Society of Automotive Engineers}{September 2018 - Present}{Team Lead - Electrical Section}{Kingston, Ontario}
\item Leading all software developers and hardware designers on the team. Assisting with technical issues in addition to defining the overall system architecture.
\item Architect of the CAN Bus network employed on all vehicles from 2020 onward. Wrote a
  custom CAN library in C++ to allow other team members to easily access CAN
  data in their respective software projects.
\item Implemented inter-MCU communication protocols using Consistent Overhead Byte Stuffing (COBS) and checksums to ensure data integrity.
\item Moved all microcontrollers from Arduino and off-the-shelf shields to fully custom STM32 ARM PCB designs.
\item Currently leading the development of an electrical vehicle for the 2023-2024 racing season.

\end{rSubsection}

\begin{rSubsection}{Youth Assisting Youth}{September 2021 - Present}{Volunteer - STEM Tutor}{Toronto, Ontario}
\item Tutoring at-risk youth in STEM subjects with a particular focus on computer science.
\item Comitted to emphasizing the value of a university education and engineering in particular.
\end{rSubsection}
%------------------------------------------------

\begin{rSubsection}{Royal St. George's College}{February 2016 - March
    2018}{Creator - iOS Schedule App}{Toronto, Ontario}
\item Created an iOS and Android scheduling app which interprets the school’s day cycle and
  allows students to view current or future days of their timetable on a calendar.
\item Developed a server side backend which holds user accounts and associated schedule data
  using Python Django and the Django REST framework and allows the creation of a public
  API.
\item Released a simplified iOS version of the app on the Apple app store in March 2018 to
  coincide with the roll out of a new schedule being tested by the school.

\end{rSubsection}
% %------------------------------------------------

\begin{rSubsection}{Royal St. George's College}{September 2017 - June
    2018}{Teaching Assistant}{Toronto, Ontario}
\item One of a select group of Grade 12 high achieving students awarded the position.
\item Responsibilities included assisting/mentoring Grade 10, 11 and 12 students with projects in
  the engineering lab, as well as completing in-house lab projects, managing the lab and
  supervising equipment use outside of school hours.
\end{rSubsection}

\end{rSection}

\end{document}
